%%%%%%%%%%%%%%%%%%%%%%%%%%%%%%%%%%%%%%%%%%%%%%%%%%%%%%%%%%%%%%%%%%%%%
% PREAMBLE
%%%%%%%%%%%%%%%%%%%%%%%%%%%%%%%%%%%%%%%%%%%%%%%%%%%%%%%%%%%%%%%%%%%%%

%% Start with one of the following:
% DOUBLE-SPACED VERSION FOR SUBMISSION TO THE AMS
\documentclass{ametsoc}

% TWO-COLUMN JOURNAL PAGE LAYOUT---FOR AUTHOR USE ONLY
%\documentclass[twocol]{ametsoc}

%%%%%%%%%%%%%%%%%%%%%%%%%%%%%%%%
%%% To be entered only if twocol option is used

\journal{jpo}

\usepackage[utf8]{inputenc}
\usepackage[T1]{fontenc}
\usepackage[french]{babel}

\newcommand{\jg}[1]{\textcolor{red}{JG: #1}}
\newcommand{\amt}[1]{\textcolor{blue}{AMT: #1}}
\newcommand{\mlc}[1]{\textcolor{green}{MLC: #1}}

%  Please choose a journal abbreviation to use above from the following list:
% 
%   jamc     (Journal of Applied Meteorology and Climatology)
%   jtech     (Journal of Atmospheric and Oceanic Technology)
%   jhm      (Journal of Hydrometeorology)
%   jpo     (Journal of Physical Oceanography)
%   jas      (Journal of Atmospheric Sciences)	
%   jcli      (Journal of Climate)
%   mwr      (Monthly Weather Review)
%   wcas      (Weather, Climate, and Society)
%   waf       (Weather and Forecasting)
%   bams (Bulletin of the American Meteorological Society)
%   ei    (Earth Interactions)

%%%%%%%%%%%%%%%%%%%%%%%%%%%%%%%%
%Citations should be of the form ``author year''  not ``author, year''
\bibpunct{(}{)}{;}{a}{}{,}

%%%%%%%%%%%%%%%%%%%%%%%%%%%%%%%%

%%% To be entered by author:

%% May use \\ to break lines in title:

\title{Barotropic vorticity balance of the North Atlantic subpolar gyre in an eddy-resolving model}

%%% Enter authors' names, as you see in this example:
%%% Use \correspondingauthor{} and \thanks{Current Affiliation:...}
%%% immediately following the appropriate author.
%%%
%%% Note that the \correspondingauthor{} command is NECESSARY.
%%% The \thanks{} commands are OPTIONAL.

    %\authors{Author One\correspondingauthor{Author One, 
    % American Meteorological Society, 
    % 45 Beacon St., Boston, MA 02108.}
% and Author Two\thanks{Current affiliation: American Meteorological Society, 
    % 45 Beacon St., Boston, MA 02108.}}

\authors{M. Le Corre\correspondingauthor{Laboratoire d'Oc\'eanographie Physique et Spatiale (LOPS), IUEM, Brest, France.}, J. Gula, and A. M. Tr\'eguier}

%% Follow this form:
    % \affiliation{American Meteorological Society, 
    % Boston, Massachusetts.}

\affiliation{Univ. Brest, CNRS, IRD, Ifremer, Laboratoire d'Oc\'eanographie Physique et Spatiale (LOPS), IUEM, Brest, France}

%% Follow this form:
    %\email{latex@ametsoc.org}

\email{mathieu.lecorre@univ-brest.fr}

%% If appropriate, add additional authors, different affiliations:
    %\extraauthor{Extra Author}
    %\extraaffil{Affiliation, City, State/Province, Country}

  
%\extraauthor{}
%\extraaffil{}

%% May repeat for a additional authors/affiliations:

%\extraauthor{}
%\extraaffil{}

%%%%%%%%%%%%%%%%%%%%%%%%%%%%%%%%%%%%%%%%%%%%%%%%%%%%%%%%%%%%%%%%%%%%%
% ABSTRACT
%
% Enter your abstract here
% Abstracts should not exceed 250 words in length!
%
% For BAMS authors only: If your article requires a Capsule Summary, please place the capsule text at the end of your abstract
% and identify it as the capsule. Example: This is the end of the abstract. (Capsule Summary) This is the capsule summary. 

%\abstract{In this study, we investigate the barotropic vorticity balance of the North Atlantic Subpolar Gyre in a high resolution simulation ($\approx$ 2-km) with a $\sigma$-level coordinate model. From what we know of the Subpolar Gyre is mostly in topographic Sverdrup dynamic where the first order balance is done between the Planetary Vorticity (PV) and the Bottom Pressure Torque (BPT) with the Non Linear Advection (NLA) term barely influencing this balance. Our finding highlight that, locally, at high resolution, the first order balance is between the BPT and the NLA while the bottom drag curl and the PV are one order of magnitude lower. Also, one part of the barotropic vorticity is injected through the BPT near the slopes along Eastern Boundary Current  and is advected toward the inside of the gyre through eddy effects while the remaining part is brought by mean baroclinic in the North Western Corner.}

\abstract{The circulation in the North Atlantic Subpolar gyre is complex and strongly influenced by the topography. The gyre dynamics is traditionally understood as the result of a topographic Sverdrup balance, which corresponds to a first order balance between the planetary vorticity advection, the bottom pressure torque and the wind stress curl. However, this dynamics has been studied mostly with non-eddy-resolving models and a crude representation of the bottom topography. Here we revisit the barotropic vorticity balance of the North Atlantic Subpolar gyre using a high resolution simulation ($\approx$ 2-km) with  topography-following vertical coordinates to better represent the mesoscale turbulence and flow-topography interactions. Our findings highlight that, locally, there is a first order balance  between the bottom pressure torque and the nonlinear terms, albeit with a high degree of cancellation between each other. However, balances integrated over different regions of the gyre -- shelf, slope and interior -- still highlight the important role played by nonlinearities and the bottom drag curls. In particular the topographic Sverdrup balance cannot describe the dynamics in the interior of the gyre. The main sources of cyclonic vorticity are the non linear terms due to eddies generated along eastern boundary currents and the time-mean nonlinear terms from the Northwest Corner. Our results suggest that a good representation of the mesoscale activity along with a good positioning of the Northwest corner are two important conditions for a better representation of the circulation in the North Atlantic Subpolar Gyre.}


\begin{document}


\maketitle

\section{Introduction}

% NOTES


% The subpolar gyre (motivation + brief description)
% eddies in the subpolar gyre
% Fan et al, 13: eddies in the Irminger Sea
% Zhao et al, 18 (x2), eddies in the Iceland basin (see ref. inside) + impacts for heat transport variability

% Structure of the mean currents in models and imporvements with resolution ( % % Treguier et al, 04
% Marzocchi et al, jms, 2015
% Danek et al jpo, 19


% Currents with strong barotropic components and strong topographically constrained currents = Barotropic vorticity balance in the subpolar gyre!
% Impact of topography: topographic sverdrup balance (Hughes, 00, Hughes & DeCuevas, 01, Hughes et al, 05, Jackson, Hughes & Williams, 06, Spence et al, 11)
%Impact of baroclinicity: How barotropic are the currents? Do the currents mostly follow f/H contours or does JEBAR play a role? (Yeager, 15) + correlation AMOC SPG vorticity
%Impact of non-linearities: What are the role of the eddy momentum fluxes? (Wang et al, 17)
% see: Sonnewald et al, 19
%%%%%%%%%%%%%%%%%%%%%%%%%%

The North Atlantic Subpolar Gyre (SPG) is a key region for the meridional overturning circulation (MOC). There, the North Atlantic surface waters coming from the subtropical gyre are transformed into denser waters that flow southward and form the lower limb of the MOC. The dynamics of the currents in the SPG is a result of strong buoyancy gradients, intense surface buoyancy and wind forcings, and exchanges of waters with the Nordic Seas through overflows. Understanding this complex dynamics is essential to better understand the mechanisms that drives the variability of the MOC. 

 The dynamics of wind-driven oceanic gyres is traditionally understood as the result of two distinct balances for the interior of the gyre and the boundary of the gyre, where currents flow along topography. In the interior, the flow follows a Sverdrup balance, which corresponds to a first order balance between the wind stress curl and a meridional transport in the barotropic (depth-integrated) vorticity balance. This balance has been shown to hold in the interior of subtropical gyres \citep{hughes2001, TDBAJS14, yeager2015, schoonover2016, sonnewald2019, LBST19}. Where the currents interact with the topography, another term becomes first order in the barotropic vorticity balance: the Bottom Pressure Torque (BPT). The BPT includes the impacts of the bottom topography on the barotropic currents, and derives from the interaction of the abyssal geostrophic flow with the sloping bottom bathymetry. Works by \citet{hughes2000,hughes2001, jackson2006, schoonover2016} have demonstrated the prevalence of the BPT in the global barotropic vorticity balance. They have shown in particular that the BPT is the dominant term in western boundary currents, thus demonstrating that viscous effects were not required to close the vorticity budget of the gyres as hypothesized in the classical works of \citet{Munk1950}. 
 The SPG circulation is strongly shaped by the bottom topography. Due to the low stratification, the currents have a strong barotropic component \citep{vanaken1995,daniault2016,fischer2004}.  They are thus strongly impacted by the steep topography around the gyre. The importance of the bottom topography in driving the SPG dynamics emerged quite early in the works of \citet{luyten1985} and \citet{wunsch1985}. The prevalence of the BPT in the SPG has also been demonstrated by \citet{hughes2001, spence2012, yeager2015}. All studies also pointed out a failure of the flat bottom Sverdrup balance in this area. 

The studies putting forward the importance of the BPT in the SPG have been using coarse resolution models. But currents in the SPG are also strongly influenced by eddies, which can modify the mean flow structure \citep{MW08}. Models then require resolutions able to resolve these effects. Eddy-permitting resolutions have been shown to improve the characteristics of the boundary currents of the SPG, including a better position of the currents, narrower lateral extensions and velocity amplitudes closer to observations \citep{treguier2005,danek2019}. The vertical structure of the currents is also improved with a more barotropic structure for the boundary currents around the SPG \citep{marzocchi2015}. These changes, compared to coarser resolution models, allow the inertial effects to become more important and modify the interactions with the topography. Also, at higher resolution, the viscosity is reduced and the bottom topography and inertial effects become prevalent, allowing the flow to better match the observations \citep{spence2012, schoonover2016}. 

Recently, \citet{sonnewald2019} clustered regions dominated by different barotropic vorticity balances using a global $1^{\circ} \times 1^{\circ}$ model. They retrieved the results of a SPG dominated by BPT effects, but also a part of the gyre dominated by Non-Linear (NL) effects, despite the relatively coarse resolution of the model. \citet{yeager2015} compared results from a $1^{\circ}$ resolution model with an eddy-permitting $1/10^{\circ}$ resolution model, and noticed an increase of the amplitude of the NL term by a factor 3 in some locations. However, it did not modify significantly the first order equilibrium between the wind, planetary vorticity and BPT.  The impact of the NL term becomes clearer at higher resolution. With a 1/20$^{\circ}$ resolution simulation \citet{wang2017} showed the importance of this term in the dynamics of recirculation gyres such as the Gulf Stream recirculation gyres or the North Western Corner. 

In addition to the horizontal resolution, the representation of the bottom topography has an impact on the structure of the flow. z-level coordinates have the tendency to create too shallow flows compared to partial step coordinates \citep{pacanowski1998}. Terrain following coordinates ( $\sigma$-level) have proven effective in representing boundary currents \citep{schoonover2016,ezer2016}. The z-level types coordinates tend to have too much viscosity and/or diffusivity close to the topography due to the presence of vertical walls. This effect is corrected when increasing the vertical resolution or using partial steps to converge to results obtained with $\sigma$-coordinates \citep{ezer2004}.%try to put bottom drag somewhere


The aim of this paper is to investigate the dynamics of the SPG by analysing the barotropic vorticity balance in a truly eddy-resolving $\sigma$-level coordinate model. To our knowledge no study of the SPG dynamics has ever been conducted at this resolution with this kind of vertical coordinates. The switch in vertical coordinates combined with eddy-resolving resolution might help to resolve smaller scale processes and allow a better representation of  flow-topography interactions overall. The paper is organised as follows: The simulation setup is presented in section 2. The mean currents characteristics and variability in the simulation are confronted to observations in section 3. The barotropic vorticity balance is analyzed for the full SPG in section 4. The balances corresponding to the different parts of the gyre are further described in section 5. To better understand what is hidden inside the non linear term we analyze it in more details in section 6. Conclusions are presented and discussed in section 7.       


%%%

%%%%



\section{Model and set-up}
%\subsection{Model setting}


To investigate the impact of the topography on the circulation, it is essential to have a good representation of the flow-topography interactions. To do so, we use a terrain-following coordinate model: the Regional Oceanic Modelling System (ROMS, \citet{shchepetkin2009}) in its CROCO (Coastal and Regional Ocean Community) version \citep{debreu2012}. It solves the hydrostatic primitive equations for velocity, temperature and salinity, using a full equation of state for seawater \citep{shchepetkin2009,shchepetkin2011}. 

To achieve a kilometric resolution at a reasonable cost, we use a one way nesting approach by defining two successive horizontal grids with resolutions $\Delta x \approx 6$ km for the parent grid covering the North Atlantic ocean (NATL) and $\Delta x \approx 2$ km for the child grid covering the SPG (POLGYR). The parent North Atlantic domain is identical to the one in \cite{renault2016}. It has 1152 $\times$ 1059 points  with a horizontal resolution of 6---7 km. The child grid has 2000 $\times$ 1600 points and a horizontal resolution of 2 km. It allows the simulation to be truly eddy resolving in most of the area, as the Rossby deformation radius varies between 10 and 20 km over the region \citep{chelton1998}. The domains are shown in figure \ref{domain}. 

The bathymetry for both domains is constructed from the SRTM30 PLUS dataset (available online at \url{http://topex.ucsd.edu/WWW_html/srtm30_plus.html}) based on the 1 min \cite{sandwell1997} global dataset and higher resolution data where available. A Gaussian smoothing kernel with a width 4 times the topographic grid spacing is used to avoid aliasing whenever the topographic data are available at higher resolution than the computational grid and to ensure the smoothness of the topography at the grid scale. Also, to avoid pressure gradient errors induced by terrain-following coordinates in shallow regions with steep bathymetric slopes \citep{beckmann1993}, we locally smooth the bottom topography $h$ to ensure that the steepness of the topography does not exceed $r=0.2$, where the local $r$-factor is defined in the x and y directions by $r_x = \frac{h(i,j)-h(i-1,j)}{h(i,j)+h(i-1,j)}$ and $r_y = \frac{h(i,j)-h(i,j-1)}{h(i,j)+h(i,j-1)}$, (i,j) representing the grid index.


Initial and lateral boundary data for the largest domain are taken from the Simple Ocean Data Assimilation (SODA, \citet{carton2008}). The NATL simulation is run from January 1st, 1999 to December 31st, 2009. It is spun up for 2 years, and the following 8 years are used to generate boundary conditions for the child grid. Our focus is the barotropic vorticity dynamics, characterized by time scales on the order of months, such that a year of spin up is sufficient for the kinetic energy to reach a state of quasi-equilibrium in POLGYR (not shown). The study is carried on the 7 remaining years between 2002 and 2009. The surface forcings are daily ERA-INTERIM data for the parent grid and 12-hourly ERA-INTERIM data for the child grid. %A weak salinity correction flux is used at the surface of the domain to account for missing freshwater fluxes related to river runoffs in the western part of the domain. It is a relaxation toward the ISAS13 climatology \citep{gaillard2016} with a time scale of 20 days.

The North Atlantic and subpolar gyre simulations have 50 and 80 vertical levels, respectively. Vertical levels are stretched at the surface and bottom \citep{lemarie2012} to have a better representation of the surface layer dynamics at the top and flow-topography interactions at the bottom. The depth of the transition between flat z levels and terrain-following $\sigma$ levels is $h_{cline} = 300$ m. The two parameters controlling the bottom and surface refinement of the grid are $\sigma _b=2$, $\sigma _s=7$ for the parent grid and $\sigma _b=3$, $\sigma _s=6$ for the child grid, corresponding to strongly stretched levels at the surface and bottom (Fig. \ref{vertical_level}).
%amt this says nothing to non-ROMS users like me. You could add (choosing whatever depth you like as an example, 3000m or other)
%In a water column of 3000m depth, this gives a layer thickness of X m (XX m) at the surface and Ym (YY m) at the bottom, for the parent (child) grid, respectively. 
The vertical mixing of tracers and momentum is done by a k-$\epsilon$ model (GLS, \citet{umlauf2003}). The effect of bottom friction is parameterized through a logarithmic law of the wall with a roughness length $Z_{0} = 0.01$ m.


\section{Mean Currents and variability}
\subsection{Mean circulation}
%\jg{Montrer la streamfunction quelque part?}

Before investigating what is driving the SPG dynamics, we first need to validate the mean circulation in our simulations. Mean velocities from the two simulations (NATL and POLGYR) at the surface and 1000-m depth are shown in figure \ref{vel}. We present at the bottom of figure \ref{vel} the amplitudes of the currents from the NOAA drifter climatology \citep{laurindo2017} at the surface and from the ARGO-based ANDRO dataset at 1000-m depth \citep{ollitrault2013,lebedev2007}. The ANDRO data have been binned on a $0.25^{\circ}\times 0.25^{\circ}$ grid and cells with less than 10 data points have been removed. 

The North Atlantic Current (NAC) represents a boundary between the subtropical and the subpolar gyres. Oceanic models have difficulties in reproducing its dynamics and particularly its Northern extension known as the NorthWest Corner \citep{bryan2007,hecht2008,drews2015}, which is centered at 50$^{\circ}$ N, 48 $^{\circ}$ W \citep{lazier1994}. These difficulties lead to the apparition of the so called "cold-bias", which can reach up to 10 $^{\circ}$C \citep{griffies2009,talandier2014, drews2015}, and which plays a role in the Atlantic low frequency variability \citep{drews2017}. The NorthWest Corner is well reproduced in our simulations, and the temperature bias at this location is less than a degree.  


After turning eastward, the NAC splits into three branches, which are strongly constrained by topography \citep{bower2008}. They cross the Mid Atlantic Ridge (MAR) through three deep fracture zones: the Charlie-Gibbs Fracture Zone (CGFZ, 52.5 $^{\circ}$ N), the Faraday fracture zone (50$^{\circ}$ N) and the Maxwell fracture zone (48$^{\circ}$ N) \citep{bower2002}. In both surface and 1000 m observations (figure \ref{vel}), the Northern branch of the NAC is more intense and corresponds to the main pathway across the MAR. The three branches are well represented in the simulations with, at the surface, an overestimation of the southern branch and an underestimation of the northern branch. At depth, ANDRO data depict an intense branch crossing the MAR at the CGFZ while the amplitude of the two Southern branches is smaller. This feature might be related to the Labrador Sea Water passing into the Eastern Basin through the CGFZ in this depth range, while in the Faraday and Maxwell Fracture zones the flow is more surface intensified. The circulation in POLGYR is closer to the observations with a better representation of the flow in the CGFZ at 1000 m. 

After crossing the MAR, the three branches head North with the two Northern ones feeding the interior of the Iceland basin and the Rockall Trough (RT) \citep{daniault2016}. The water coming from the Maxwell fracture zone recirculates southward in the West European Basin \citep{paillet1997}. As most of the models \citep{treguier2005,deshayes2007}, NATL and POLGYR are consistent with observations for the circulation in the Eastern Basin  with a good positioning of the two main branches passing respectively in the Maury channel (deepest part of the Iceland Basin west of Hatton Bank) and the RT. 

A deep permanent anticyclonic eddy is found in the Rockall Trough \citep{fisher2018,smilenova,lecorre2019}. This structure is detectable in the ANDRO dataset around 55$^{\circ}$ N, 12$^{\circ}$ W (figure \ref{vel}). It is not present in NATL while it appears in POLGYR, albeit with too intense velocities. In NATL at depth, there is a strong southward flow in the western part of the RT due to the wrong representation of the Faroe Bank channel. As the topography is strongly smoothed, the channel is not properly represented and does not allow the dense water coming from the Nordic Seas to pass through it and feed the Iceland Scotland Overflow Water properly \citep{hansen2016,kanzow2014}. Thus, the water is recirculating in the western part of the RT, creating a spurious pattern (figure \ref{vel}). The problem is solved by increasing the horizontal resolution and improving the representation of the topography, which corresponds to a wider opening of the channel and allows a more realistic circulation in the RT. 


Further north, part of the flow continues to the Nordic Seas \citep{rossby2012}, while the other part follows the Reykjanes Ridge (RR). A common bias in models east of RR is a too intense southward flow at the surface \citep{treguier2005}. This bias is present in NATL but disappears at higher resolution in POLGYR, which is closer to the circulation observed by the drifters. On the western side of the RR the signal of the strong northward Irminger current visible in observations is well resolved by the simulations (figure \ref{vel}). 

At 1000-m depth, Argo floats reveal a continuous current following the Eastern RR flank until reaching the CGFZ, with some of the flow crossing the ridge North of 57.3$^{\circ}$ N and some crossing at the Bight Fracture Zone (56-57$^{\circ}$ N). This is coherent with the results from \citet{petit2018}, which observed that water at this depth (their layer 3) was more likely to cross the ridge North of 56$^{\circ}$N. This southwestward flow is present in our simulations, with too intense velocity amplitudes in NATL, but realistic amplitudes at higher resolution in POLGYR. In both cases, we clearly see the flow crossing the ridge north of 56$^{\circ}$N. On the western side of RR, the velocity in the simulations is too strong compared to observations.
The mean subpolar gyre intensity in the model (Fig. \ref{psil}), computed as the cumulative transport from Iceland to 53.15$^{\circ}$ N along the crest of the RR, is equal to -25 Sv and compares well with the -21.9 $\pm$ 2.5 Sv monthly average in \citet{petit2018}.  

Numerous recirculations are present in the SPG, many of them occurring near the intense boundary currents along Greenland and around the Labrador sea \citep{reverdin2003, flatau2003,cuny2002}. The recirculation cells are present in the Labrador sea \citep{lavender2000,cuny2002} and extend to the Irminger basin \citep{Holliday2009}. Theses features are mainly driven by the topography and the wind as described in \citet{kase2001,spall2003}, and are stable in time \citep{palter2016}. Some models are unable to reproduce correctly the recirculation cells, especially the one in the center Labrador Sea \citep{treguier2005}. In our case, this recirculation is well represented (figure \ref{vel}). The counter current flows offshore the Labrador continental slope, with a North extension at 60$^{\circ}$ N, which matches observations from \citet{lavender2005}.  At the tip of Greenland, this counter current separates in two to form a branch flowing inside the Irminger Basin while the other branch is redirected southward. This second branch is relatively intense in our simulation but is also present in ANDRO data (\cite{fischer2018}, their figures 3 and 5a). % and links the northern branch of the NAC. 

\subsection{The mesoscale activity}

The mesoscale activity plays a big role in redistributing water masses properties in the SPG \citep{dejong2016,zhao2018b}. The presence of mesoscale eddies can be inferred by their signatures on the Eddy Kinetic Energy (EKE). From the surface EKE signal extracted from NOAA drifters on a $0.25^{\circ}\times 0.25^{\circ}$ grid (Fig. \ref{energy}), we retrieve the main hot spots described by \citet{flatau2003} in the SPG: the Labrador Sea, the Irminger and Iceland basins. Those signals are mainly due to generation of mesoscale eddies through baroclinic and barotropic instabilities of the boundary currents.

EKE amplitudes in the NATL simulation are weaker than in observations, but the eddy activity is enhanced when the resolution is increased. The POLGYR simulation displays similar EKE patterns than observational data in every basins (Labrador, Irminger and Iceland) with close amplitudes over most of the SPG. The EKE patterns corresponding to the generation of Irminger Rings have higher magnitudes in POLGYR than in the NOAA drifters data.

A way to quantify the mesoscale activity at depth is to look at the vertical isopycnal displacements. When referenced to a mean, it represents the Eddy Available Potential Energy (EAPE) or the amount of energy stored in the potential energy reservoir due to mesoscale activity \citep{lorenz1955}. This quantity is a proxy of the baroclinic activity in the interior of the ocean. We compare EAPE from the simulations with the atlas of \citet{roullet2014} constructed from Argo data (Fig. \ref{energy}). In NATL (at 6 km resolution) most of the baroclinic activity already seems well resolved. However, observations highlight an EAPE maximum on the western flank of the RR that is missing in NATL, but appears only in POLGYR (at 2 km resolution). On the contrary, strong patches of EAPE are visible along the boundary currents of the western half of the SPG in NATL, but are not visible in observations. Interestingly these patterns weaken in POLGYR, potentially pointing to a change in the vertical structure of the currents at higher resolution. Another factor to take into consideration is the lack of Argo measurements close to the boundaries, which might cause an underestimation of EAPE at these locations.


\section{Vorticity balance of the subpolar gyre at high resolution}

\subsection{An overall view of the subpolar gyre vorticity balance}

%To study the dynamics of the gyre we choose to compute the barotropic vorticity balance. 
The barotropic vorticity equation is obtained by integrating the momentum equations in the vertical and cross-differentiating them \citep{gula2015}:

$$\underbrace{\frac{\partial \Omega}{\partial t}}_{rate} = -\underbrace{\mathbf{\nabla}.(f \mathbf{\overline{u}})}_{planet.\;vort.\;adv}+\underbrace{\frac{J(P_b,h)}{\rho _0}}_{BPT} +\underbrace{\mathbf{k}.\mathbf{\nabla} \times \frac{\mathbf{\tau} _{wind}}{\rho_{0}}}_{wind\;curl} -\underbrace{\mathbf{k} \cdot \mathbf{\nabla} \times \frac{\mathbf{\tau} _{bot}}{\rho_{0}}}_{BDC} +\underbrace{D_{\Sigma}}_{horiz.\;diffusion}+\underbrace{A_{\Sigma}}_{NLA}$$

where the vorticity $\Omega$ is the curl of the vertically integrated components of the velocity between the bottom and the surface: $\Omega = \mathbf{k} \cdot \mathbf{\nabla} \times \overline{\mathbf{u}}$, with $\mathbf{u} = (u,v)$ the velocities in the $(x,y)$ direction. The overbar defines a vertically integrated quantity:

$$\overline{u}=\int^{\eta}_{-h} u \; dz$$

with $\eta(x,y,t)$ the free surface height and $h(x,y)$ the topography. It is possible to decompose the planetary vorticity advection $-\nabla.(f\overline{u})=-\beta V-f \frac{\partial \eta}{\partial t}\approx - \beta V$, with $V$ the vertically integrated meridional component of velocity, if we consider a mean over a long enough time period such that $\frac{\partial \eta}{\partial t} \approx 0$.

The non linear term can be written as:
%
$$A_{\Sigma}= -\frac{\partial ^2 (\overline{vv}-\overline{uu})}{\partial x \partial y}-\frac{\partial ^2 \overline{uv}}{\partial x \partial x} +\frac{\partial ^2 \overline{uv}}{\partial y \partial y}.$$
%
The expression for $A_{\Sigma}$ is similar to the one shown in \citet{schoonover2016} (their equation (2)) but in our case, the integration between -h and $\eta$ allows their last term to cancel out with a residue from the inversion of the time derivative and the vertical integral in the rate term. The bottom pressure torque J(P$_b$,h) is the Jacobian of the bottom pressure and the depth of the topography. It encompasses the effects of the varying topography on the flow, and is known to play a key role in the barotropic vorticity balance of the SPG. In an idealized case of a geostrophic current flowing along a topography in free-slip condition, the BPT can be written $\frac{J(P_b,h)}{\rho _0}=f u_b \cdot \nabla h$ where $\rho_0$ is the mean reference density and the subscript $b$ denotes a field at the bottom. Given the kinematic condition at the bottom: $-u_b \cdot \nabla h =w_b$, the BPT can be written $\frac{J(P_b,h)}{\rho _0}=-fw_b$, which highlights the relation between the BPT and vortex stretching when the flow crosses an isobath.

The barotropic vorticity terms have already been computed for the North Atlantic using different models (OCCAM, ECCO, UVic ESCM, POP) at different resolutions ($1.8^{\circ}\times3.6^{\circ}$, $1^{\circ}$, $0.25^{\circ}$, $0.2^{\circ}\times 0.4^{\circ}$, $0.1^{\circ}$) in \citet{hughes2001}, \citet{spence2012}, \citet{sonnewald2019}, and \citet{yeager2015}. Their major result is that the barotropic vorticity balance in the subtropical and subpolar gyres is at first order a balance between $\beta$V, $\nabla \times \frac{\tau _{wind}}{\rho_{0}}$, and $\frac{J(P_b,h)}{\rho _0}$. 
%amt: what do you make of the Wang et al (2017) paper? They estimate the terms in a way different from others; they use NEMO at 1/20° resolution.

In the subtropical gyre, the barotropic vorticity balance is close to a Sverdrup balance away from the boundaries ($\beta V \approx \nabla \times \frac{\tau _{wind}}{\rho_{0}}$), while the closure of the northward branch of the gyre at the western boundary is done primarily through BPT ($\beta V \approx \frac{J(P_b,h)}{\rho _0}$)  \citep{schoonover2016}.

The barotropic vorticity balance in the SPG is slightly more complex due to the strong impact of the topography. Along the northern and western boundaries of the SPG, the first order balance is between meridional advection and BPT ($\beta V \approx \frac{J(P_b,h)}{\rho _0}$) (\textit{e.g.} \citet{hughes2001}, their Fig. 4; \citet{yeager2015}, their Fig. 1), with a significant impact of the wind only in the northern part of the gyre along the Greenland coast. When the resolution of the model is increased from $1^{\circ}$ to $0.1^{\circ}$ in \citet{yeager2015}, the main balances stay qualitatively similar, showing a modest effect of the eddies. Using a shallow water model with higher resolution ($1/20^{\circ}$), \cite{wang2017} illustrates the importance of the NL term in the dynamics of specific regions such as the Gulf Stream and the recirculation gyres. The viscous torque decreases in the boundary currents due to the lower viscosity of their model.

\subsection{Spatial scales of the vorticity balance}

In our simulations, the BPT balances the advection of vorticity at leading order everywhere in the domain  (Fig. \ref{BPT_NL_unsmoothed}). This is qualitatively different from the vorticity balances shown in \citet{yeager2015}, but it is similar to the results of \citet{gula2015} in the Gulf Stream region with the same ocean model and a similar horizontal resolution. This highlights the fact that locally the flow is able to follow isobaths due to an equilibrium between the NL term (making the flow cross isobaths) and the bottom pressure anomaly.
%amt I do not understand the following sentence. Can you explain better?  

%This highlights the fact that the bottom pressure anomaly balances the advection term in the momentum equation as the flow attempts to follow isobaths.
Both terms exhibit small scales related to topographic features, but with a high degree of cancellation between each other. The sum of the BPT and NL terms (Fig. \ref{BPT_NL_unsmoothed}c) is often an order of magnitude smaller than the amplitude of the terms considered individually and exhibits patterns and amplitudes matching the advection of planetary vorticity. This cancellation is also clear in \cite{wang2017}, their Fig. 3, where the transport driven by mean flow advection balances the one driven by the BPT, both having amplitudes larger than the wind stress curl-driven transport.

% why say "Another"? it is not clear that you have presented above "a way to get rid of the small....". I would prefer to avoid "get rid off", this is spoken language not very appropriate in this paper. Moreover, you could say that smoothing is necessary and give references (what kind of smoothing did the other authors)?
To facilitate the interpretation of maps of NL and BPT terms, the impact of small topographic scales has to be reduced by smoothing with a large enough length scale. 
NL terms in particular are expected to be smoothed out on scales larger than 1-2$^{\circ}$ \citep{hughes2001}. Fig. \ref{spatial_BV} shows all terms smoothed with a gaussian kernel of 1$^{\circ}$ radius. Even with such smoothing, the BPT and NL terms are still significantly larger than the corresponding results from the $0.1^{\circ}$ simulation of \citet{yeager2015}. However, their sum $\frac{J(P_b,h)}{\rho _0}+A_{\Sigma}$ (\ref{spatial_BV} (f))is of the same order of magnitude than the $\beta$V (Fig. \ref{spatial_BV} (a)) and the Bottom Drag Curl (BDC, Fig. \ref{spatial_BV} (e)). 

The curl of the wind stress in POLGYR has the same pattern and amplitude than in \citet{yeager2015}. It is mostly positive with the strongest signal on the Eastern coast of Greenland. The amplitude of the $\beta$V term is slightly stronger in our model than in coarser resolution simulations. In the simulations of \citet{hughes2001} and \citet{yeager2015}, the patterns of the $\beta$V term seems to indicate much wider currents. Here, the patterns correspond to thinner and more intense currents, closely following the continental slopes, in agreement with the observations.

%\jg{attention dans Yeager le smoothing est de 2deg.}
In our simulations, the amplitude of the viscous torque, due to the horizontal viscosity of the model ($D_{\Sigma}$), is very small, while the amplitude of the BDC is comparable to the $\beta V$. This is opposite to the results of the $0.1^{\circ}$ POP simulation of \citet{yeager2015}. In fact, their viscous term is qualitatively similar in pattern to the bottom drag curl in our simulation. The boundary conditions near the topography are quite different in the two models due to the different vertical coordinates. The z-levels coordinates have vertical walls between each level, with parameterized lateral viscosity, which explains the pattern in \citet{yeager2015}. The $\sigma$-levels coordinates have no lateral boundary conditions and friction on the topographic slopes is only parameterized as a bottom drag. The amplitude of the BDC is however stronger in our simulation than the viscous term in \citet{yeager2015} and seems to play a important role in balancing the BPT and $\beta V$ terms over the shelf and on the upper part of the continental slope along the northern and western boundaries of the gyre.

\subsection{Link between barotropic vorticity balance and bottom velocities}


As explained previously, the bottom pressure torque J(P$_b$,h) can be identified with a bottom vortex stretching term: $\frac{J(P_b,h)}{\rho _0}=f u_{gb}.\nabla h = -f w_{gb}$, where $u_{gb}$ is the horizontal geostrophic bottom flow. 

The computation of the BPT in \citet{spence2012} is performed by directly estimating the term  $-f w_{b}$, where $w_{b}$ is the vertical velocity at the bottom. However this estimation does not take into account the presence of an ageostrophic component of the velocity at the bottom, in particular the Ekman component of the velocity due to the bottom drag. The same computation in our model leads to the results of Fig. \ref{decomp_bpt}b, which are very different from the actual bottom pressure torque \ref{decomp_bpt}a. It gives results quite similar to \citet{spence2012} with positive signals - implying downwelling of bottom currents - over most of the boundaries of the gyre. But this downwelling is a result of the Ekman currents oriented to the left of the main bottom geostrophic currents, which are flowing with the shallower topography on their right around the gyre.

Following \citet{mertz1992} and \citet{yeager2015}, the BPT can be further decomposed into:
%
$$\frac{J(P_b,h)}{\rho _0}=f u_{gb}.\nabla h = \frac{f}{h} \overline{u_g}.\nabla h +  h(JEBAR),$$
which illustrates that the bottom geostrophic currents that appears in the expression of BPT are the sum of a vertically averaged part and a baroclinic part directly related to the JEBAR term. The term $ \frac{f}{h} \overline{u_g} \cdot \nabla h$ highlights regions where the depth-averaged flow is crossing isobaths, and the $h(JEBAR)$ term where the baroclinic effects are playing a role to decouple the bottom flow from the barotropic flow through the geostrophic shear. In figure \ref{decomp_bpt}c the geostrophic velocity has been computed from the thermal wind balance referenced at the surface.

Along the continental slopes, on the western and northern part of the gyre, the flow is close to barotropic and the $ \frac{f}{h} \overline{u_g} \cdot \nabla H$ term has similar patterns and amplitudes than the BPT. This contrasts with results from \citet{yeager2015}, who found that the $h(JEBAR)$ term was almost an order of magnitude larger than the BPT in these regions. However over the southern and eastern part of the gyre, it is clear that the structure of the flow is much more baroclinic and the $ \frac{f}{h} \overline{u} \cdot \nabla h$ and  $h(JEBAR)$ terms are both an order of magnitude larger than the BPT.

\section{Integrated vorticity balance for the shelf, slope and interior of the gyre}

\subsection{Gyre integrated barotropic vorticity balances}

The maps of the barotropic vorticity terms, with various degrees of smoothing, can help identify the locally dominant terms, but do not enable us to identify the important balances at the gyre scale. Spatial integrations are performed inside different gyre contours (Fig. \ref{intgyre_BV}) to better understand the main contributions to the circulation of the subpolar gyre. 

We distinguish the shelf area from the gyre using a contour of barotropic streamfunction of -3 Sv. This contour is chosen because it corresponds to the largest possible closed contour of the barotropic streamfunction. We can check that the term $-\nabla.(f\overline{u}) \approx -\beta$V integrates to zero over such a contour (Fig. \ref{intgyre_BV} c). The shelf thus defined corresponds to an area with a mean depth of 290 meter and is extending from the South of Iceland to Flemish cap (blue area in Fig. \ref{intgyre_BV} b).

When integrated inside the -3 Sv contour (which means excluding the shelf area, Fig. \ref{intgyre_BV}c), the main sources for the cyclonic circulation of the gyre are the wind and the BPT. They are balanced by the BDC. The wind input does not contribute much locally (figure \ref{spatial_BV}), but becomes significant when integrated spatially over the whole gyre. The BPT is the major source of positive vorticity and helps the flow move cyclonically around the gyre. The BDC and NL terms act as sinks of vorticity, but the NL term is much smaller than the BDC. The BDC is very intense where the current flows close to a steep topography, as in the case of the Labrador Current (LC) and the West Greenland Current. 

When integrated over the whole gyre (Fig. \ref{intgyre_BV}a), the balance is slightly different. The wind is still a major contributor for the cyclonic circulation and the BDC still represents the major sink of vorticity. However, the NL term replaces the BPT as a source of cyclonic vorticity for the gyre. In this interpretation, both the wind and the NL term forces the gyre cyclonically, while the BDC and BPT balance this input. 

The difference between the two balances is highlighted by looking at the balance in the region in-between the two contours, which covers the upper slope and the shelf. It corresponds to a balance between BPT, NL and bottom drag. This balance is close to the one described in \citet{csanady1978} and evokes a buoyancy driven flow in this area \citep{chapman1989}. Indeed, with a switch to ($n$,$s$) coordinates system with $n$ the right handed coordinates (here oriented toward shallower water) and $s$ the distance along flow, the BPT can be written $\frac{J(P_b,h)}{\rho _0}=-\frac{\partial P_b}{\partial s }\frac{\partial h}{\partial n}$. A negative value of BPT then means $\frac{\partial P_b}{\partial s }<0$ corresponding to a buoyancy driven current.

%\jg{tu fais référence à quels travaux? Il y aurait besoin d'aller un peu plus loin dans l'interpretation de cet equilibre}

\subsection{Barotropic vorticity balance in the interior of the gyre}

It is clear from the patterns of the different terms of the barotropic vorticity balance that the local balances over the boundary currents are very different than what is happening in the interior of the gyre. The classical picture of a gyre interior (far from the boundaries) in a quasi-Sverdrup balance that applies in the subtropical gyre,  does not seem to apply anywhere in the SPG. 

To better understand what drives the interior part of the subpolar gyre, we further divide the domain into an interior and a boundary part, as represented in fig. \ref{int_bound_inner_BV}. The two domains are defined using the -3 Sv line as previously, and the 3000 m isobath. What is between the -3 Sv line and the 3000 m isobath is considered as the slope region and the rest is considered as the interior area. The choice of the 3000 m isobath is somehow subjective but the results are not sensitive to the choice of a specific isobath. 

In the slope region, the main source of cyclonic vorticity is the BPT. The curl of the wind and the $\beta$V are also positive. The strongly negative NL term indicates advection of cyclonic vorticity outside of this domain toward the shelf or the gyre interior.

In the interior, the NL term represents the major contribution to the cyclonic circulation. It is balanced by the BDC, the BPT and the $\beta$V terms. Contributions from the BDC are of similar magnitude in the interior and the slope area. The wind input of vorticity is smaller than in the slope region, as the major wind source of vorticity is located along the Greenland area (fig \ref{spatial_BV} (d)) and not uniformly distributed over the gyre. It confirms that the gyre interior in not in Sverdrup balance at the first order, which would imply a dominant balance between a negative $\beta$V and a positive input from the curl of the wind stress, but is driven instead by nonlinear effects. The comparison between balances in the interior and slope regions indicates that the NL term helps to redistribute vorticity from the boundary toward the interior of the gyre. %The cyclonic vorticity is provided by the BPT at the boundaries of the gyre, and balanced by a sink of positive vorticity by the BDC all over the gyre.


\subsection{Balance in the slope area}

The main source of cyclonic vorticity inside the gyre is related to the NL term, which helps transferring the vorticity from the boundary toward the inside. But which boundary regions are the main contributors of vorticity to the interior? 

Several type of regions can be identified by looking at the dominant terms in the barotropic vorticity balance  (Fig. \ref{bv_zones}): The western boundary areas in cyan, which include the Western Labrador Sea (WLS), Eastern Greenland (EG) and Eastern Reykjanes Ridge (ERR); the eastern boundary regions in yellow, which include the Western Greenland (WG), the Western Reykjanes Ridge (ERR) and the eastern part of the Iceland Basin; and the Northwest regions in green, which include the extension of the Denmark Strait and Iceland Scotland overflows, and the northwestern part of the Labrador Sea.

The barotropic vorticity balance in the western boundary areas (cyan in Fig. \ref{bv_zones}) is close to the typical equilibrium of Western Boundary Currents (WBC) \citep{schoonover2016,gula2015} with an equilibrium between the planetary vorticity and the BPT. For the WLS, the deviation from WBC dynamics is small and is related to a bottom drag signal. We excluded the Southern part near Flemish Cap (48$^{\circ}$ N, 46$^{\circ}$ W) (not shown) where the dynamics is driven by a positive input of planetary vorticity and BPT balanced by the NL term. The case of the ERR is slightly different with  no net meridional transport in this area. The main input of vorticity is provided by the NL term, which is related to inertial effects from the current following the Iceland Shelf. In this area the input of positive vorticity is mainly balanced by topography and the drag corresponding to a local dissipation of vorticity. From this we can infer that western boundary areas do not provide cyclonic vorticity to the gyre interior.

Three regions (green in Fig. \ref{bv_zones}) have in common a dominant contribution from the bottom drag. Vertical sections of the mean along-stream current (Fig. \ref{section_vit} (a),(c),(e)) in these areas reveal strong intensified bottom current (especially near the Iceland shelf and the Denmark Strait). In comparison, WBCs have a more surface intensified structure with reduced amplitudes near the bottom (Fig. \ref{section_vit} (b),(d),(f)). In Fig.\ref{section_vit}, vorticity balances are indicated. They differ from Fig.~\ref{bv_zones} because the integration is restricted to the boundary current, excluding recirculations. In Fig.~\ref{section_vit} (a),(c),(e) the BPT amplitudes are reduced (and even change sign) compared to Fig.~\ref{bv_zones}. This reflects the sensitivity of the vorticity balance to the location of the boundary on the continental slope. The -3~Sv contour used in Fig.~\ref{bv_zones} does not coincide everywhere with the top of the continental slope used in Fig.\ref{section_vit}. 
% The choice of the area of integration is to be incriminated for this as we choose the shelf break as a lateral boundary. This change does not seem to have any impact on the dynamics of WBCs but imply that the topography is not of significant importance for providing vorticity in near overflows regions. 
The dynamics in the extension of the Denmark Strait and Iceland Scotland overflows is a balance between the NL term and BDC, while in the Northwestern Labrador sea, the BDC balances the $\beta$-effect. As the BDC is the main sink of vorticity and only acts locally, no advection of positive vorticity toward the inside of the gyre can come from these locations.
%\jg{Je trouve ça bizarre de mettre le courant du Labrador dans une catégorie outflow. C'est interessant de décrire que sur cette partie on ne retrouve pas un équilibre type WBC, mais plutôt un équilibre entre beta V et friction, mais je ne l'appelerais pas outflow pour autant.}

In Eastern boundary regions (yellow in Fig. \ref{bv_zones}), most of the cyclonic vorticity is provided by flow-topography interactions through the BPT and is balanced by the NL term. These regions are located where a strong eddy activity is observed (Fig. \ref{energy}), which might be responsible for the high amplitude of the NL term. This negative NL signal implies an export of positive vorticity toward either the shelf or the gyre interior. 


\section{Characterisation of the nonlinear term}

%amt in the first sentences of this paragraph, you could make clearer for the reader why it is important to do what you do. Perhaps the NL term in your analysis is larger than in previously published papers (?) and thus deserves an in-depth investigation; or remind the reader of the novelty of your analysis (sigma coordinates, very high resolution)
The NL term is locally important and balances the bottom pressure torque at small scales (Fig.\ref{BPT_NL_unsmoothed}). When integrated over the gyre it plays a role in exporting cyclonic vorticity from the boundary toward the interior of the gyre. The NL term is however quite difficult to interpret as  many processes are hidden inside the vertical and time integrals. 

By decomposing the velocity in a barotropic and baroclinic part ($u = \overline{u} + u'$) the NL advection term can be written as:
$$A_{\Sigma}=\underbrace{A(\overline{u},\overline{v})}_{A^{bt}_{\Sigma}}+\underbrace{A(u',v')}_{A^{bc}_{\Sigma}}$$
where the barotropic part can be written as $A(\overline{u},\overline{v})= \overline{u}\Omega _x +\overline{v}\Omega _y$ which is the advection of the barotropic vorticity by the barotropic flow.  

We show these terms integrated over the slope area and interior (same as Fig.\ref{int_bound_inner_BV}) in Fig. \ref{int_NL}. Over the slope area, both terms are negative and contribute to export cyclonic vorticity. The barotropic part is much larger than its baroclinic counterpart and export most of the vorticity, as can be expected from the barotropic structure of the currents over the slope. In the interior, both terms are positive, corresponding to an input of cyclonic vorticity for the interior (Fig. \ref{int_NL}), but the NL term is evenly divided between its barotropic and baroclinic contributions. The North West corner provides about half of this baroclinic NL input, while the remaining part comes mostly from the South-Eastern boundary. The exchange of barotropic vorticity is only due to the barotropic NL term between the slope region and the interior. 
% (EXPLIQUER COMMENT on le voit sur la figure)
 %Considering the need from the inner part, the boundary value is too high meaning that a large quantity of the mean NL advection term only helps locally for the balance of the boundary current. The remaining part 

%amt: has the eddy-mean decomposition been discussed in published papers? If it has been done before, you can cite these papers and say how your work is similar or different; if it has never been done before, you should say so! 
It is also possible to decompose the NL term into a time mean and eddy part by writing $u = \langle u \rangle + u^*$ where $\langle \bullet\rangle$ is the time average and the star denotes the fluctuation part. By putting this in the non linear operator $A_{\Sigma}$ we have:
$$A_{\Sigma}(u,v)=\underbrace{A_{\Sigma}(\langle u \rangle, \langle v \rangle)}_{A_{\Sigma}^{mean}} + \underbrace{A_{\Sigma}(u^*,v^*)}_{A_{\Sigma}^{eddy}} +\underbrace{\langle  2\frac{\partial ^2 \overline{\langle v \rangle v^*} -\overline{\langle u \rangle u^*}}{\partial xy} +\frac{\partial ^2 \overline{\langle u \rangle v^*} + \overline{\langle v \rangle u^*}}{\partial xx} - \frac{\partial ^2 \overline{\langle u \rangle v^*} +\overline{ \langle v \rangle u^*}}{\partial yy}\rangle}_{\varepsilon}$$
The $\varepsilon$ part is the residue of the cross product and its value is negligible compared to both the mean and eddy parts. 

When integrated over the slope area (Figure \ref{int_NL}), the eddy component dominates over the mean one. In the interior area, the supply of barotropic vorticity is also mainly due to the eddy component but the mean component contributes about a third of the total. Almost all of this mean signal is coming from the North West corner, consistent with \citet{wang2017}, while the eddy part is dominant over the rest of the interior.  %Taking into account that the NWC covers most of the need for the for the mean signal, the signal generated inside the boundary might be used for the shelf dynamics among other things. Also, vorticity advection by eddies generated along the boundary area is too large compared to what is needed by the interior (when excluding the NWC), the leftover is advecting barotropic vorticity outside of the domains. 


We can identify several processes providing cyclonic barotropic vorticity to the subpolar gyre. The most important is the eddy contribution coming from the boundary area that is associated with a barotropic contribution. Barotropic vorticity is also provided through a mean-baroclinic signal coming from the NWC. In comparison, in the lower resolution simulation (not shown) most of the vorticity is advected inside the gyre by mean-barotropic processes but the amplitude of the NL term is cut by half.


%Where the $\varepsilon$ part being the residue of the cross product and its value is negligible compared to both mean and eddy part. When integrated over the boundary area, the eddy component dominates over the mean one. In the inner part, the supply of barotropic vorticity is mainly linked to the eddy component but the mean component is also non negligible. After decomposing the inner as done for the barotropic baroclinic component, the South-western corner counts for the majority of the mean component \citep{wang2017}.     

\section{Summary and Conclusions}

We have studied the dynamics of the North Atlantic Subpolar gyre in a numerical model with, for the first time, terrain following coordinates and a mesoscale-resolving resolution ($\Delta x \approx 2$ km). The combination of the high resolution with $\sigma$-levels allows us to better resolve the effects of the mesoscale turbulence and of the complex bottom topography. The representation of the mean currents and their variability is improved compared to previous simulations with coarser resolution. In particular, the simulations produce realistic levels of mesoscale turbulence at the surface and in the interior, as seen from comparisons of eddy potential and kinetic energy with observations from Argo floats and surface drifters.

The role of the topography is essential in the SPG. This impact is reflected in the barotropic vorticity balance of the gyre through the Bottom Pressure Torque. The Bottom Pressure Torque is sometimes interpreted as the effect of the vortex stretching due to the bottom flow over topography, as expected for a predominantly geostrophic flow. However, we show here that the ageostrophic effects, in particular due to the viscous bottom drag, are predominant at the bottom and the BPT cannot be estimated from the bottom vertical velocity. %We then emphasize that computation of the bottom pressure torque can not be directly derived from the bottom vertical velocity but from the value of the pressure at the bottom. 

Barotropic vorticity balances are opposite in the shelf region compared to the inside of the gyre. The main balance in the shelf region is between a negative bottom pressure torque and a positive bottom drag, which is typical of a buoyancy driven current. Inside the gyre, the inputs of positive vorticity from the BPT and the wind curl, are balanced by the bottom drag curl. The important role played by the bottom drag and the weak role played by the viscous torque, compared to other models, is related to the choice of $\sigma$-level coordinates and high horizontal resolution. %Indeed a terrain following coordinate have no parameterization for lateral boundary explaining why we do not have viscous torque near topographic slope. 

The bottom pressure torque has a large amplitude where boundary currents flow along the steep continental slope. It is the main term  balancing the meridional transport of water in western  boundary currents, except for some regions with dense water overflows where the bottom drag curl can become predominant. In the eastern boundary currents (northward flowing), the strong input of positive vorticity by the bottom pressure torque is balanced by the nonlinear term. The nonlinearities, which are essentially due to the eddying activity, allow advection of the positive vorticity from the boundary toward the interior of the gyre. The North Western Corner is also instrumental in feeding positive vorticity to the gyre interior through its southern boundary, mostly through time-mean baroclinic fluxes.

The nonlinear term is the main forcing for the interior part of the gyre, overcoming the effects of the wind curl and bottom pressure torque. This is putting forward the failure of the classical Sverdrup balance or even of a topographic Sverdrup balance in the interior of the Subpolar gyre, and emphasizing the importance of the inertial effects to obtain a more realistic Subpolar gyre circulation.




%%%%%%%%%%%%%%%%%%%%%%%%%%%%%%%%%%%%%%%%%%%%%%%%%%%%%%%%%%%%%%%%%%%%%
% ACKNOWLEDGMENTS
%%%%%%%%%%%%%%%%%%%%%%%%%%%%%%%%%%%%%%%%%%%%%%%%%%%%%%%%%%%%%%%%%%%%%
%

%\acknowledgments
%Start acknowledgments here.

%%%%%%%%%%%%%%%%%%%%%%%%%%%%%%%%%%%%%%%%%%%%%%%%%%%%%%%%%%%%%%%%%%%%%
% APPENDIXES
%%%%%%%%%%%%%%%%%%%%%%%%%%%%%%%%%%%%%%%%%%%%%%%%%%%%%%%%%%%%%%%%%%%%%
%
% Use \appendix if there is only one appendix.
%\appendix

% Use \appendix[A], \appendix}[B], if you have multiple appendixes.
%\appendix[A]

%% Appendix title is necessary! For appendix title:
%\appendixtitle{}

%%% Appendix section numbering (note, skip \section and begin with \subsection)
% \subsection{First primary heading}

% \subsubsection{First secondary heading}

% \paragraph{First tertiary heading}

%% Important!
%\appendcaption{<appendix letter and number>}{<caption>} 
%must be used for figures and tables in appendixes, e.g.,
%
%\begin{figure}
%\noindent\includegraphics[width=19pc,angle=0]{figure01.pdf}\\
%\appendcaption{A1}{Caption here.}
%\end{figure}
%
% All appendix figures/tables should be placed in order AFTER the main figures/tables, i.e., tables, appendix tables, figures, appendix figures.
%
%%%%%%%%%%%%%%%%%%%%%%%%%%%%%%%%%%%%%%%%%%%%%%%%%%%%%%%%%%%%%%%%%%%%%
% REFERENCES
%%%%%%%%%%%%%%%%%%%%%%%%%%%%%%%%%%%%%%%%%%%%%%%%%%%%%%%%%%%%%%%%%%%%%
% Make your BibTeX bibliography by using these commands:
% \bibliographystyle{ametsoc2014}
% \bibliography{references}
\bibliographystyle{ametsoc2014}
\bibliography{vort_240419}

%%%%%%%%%%%%%%%%%%%%%%%%%%%%%%%%%%%%%%%%%%%%%%%%%%%%%%%%%%%%%%%%%%%%%
% TABLES
%%%%%%%%%%%%%%%%%%%%%%%%%%%%%%%%%%%%%%%%%%%%%%%%%%%%%%%%%%%%%%%%%%%%%
%% Enter tables at the end of the document, before figures.
%%
%
%\begin{table}[t]
%\caption{This is a sample table caption and table layout.  Enter as many tables as
%  necessary at the end of your manuscript. Table from Lorenz (1963).}\label{t1}
%\begin{center}
%\begin{tabular}{ccccrrcrc}
%\hline\hline
%$N$ & $X$ & $Y$ & $Z$\\
%\hline
% 0000 & 0000 & 0010 & 0000 \\
% 0005 & 0004 & 0012 & 0000 \\
% 0010 & 0009 & 0020 & 0000 \\
% 0015 & 0016 & 0036 & 0002 \\
% 0020 & 0030 & 0066 & 0007 \\
% 0025 & 0054 & 0115 & 0024 \\
%\hline
%\end{tabular}
%\end{center}
%\end{table}

%%%%%%%%%%%%%%%%%%%%%%%%%%%%%%%%%%%%%%%%%%%%%%%%%%%%%%%%%%%%%%%%%%%%%
% FIGURES
%%%%%%%%%%%%%%%%%%%%%%%%%%%%%%%%%%%%%%%%%%%%%%%%%%%%%%%%%%%%%%%%%%%%%
%% Enter figures at the end of the document, after tables.
%%
%






\begin{figure}[t]
\centerline{\includegraphics[width=16cm]{./fig_os/fig01.png}}
\caption{Snapshot of the relative vorticity at 500 m depth in the North Atlantic in the NATL simulation. The NATL grid ($\Delta x \approx 6$ km) covers most the North Atlantic, and the POLGYR grid (smaller rectangle, $\Delta x \approx 2$ km) covers the subpolar gyre.}
\label{domain}
\end{figure} 

\begin{figure}[t]
\centerline{\includegraphics[width=16cm]{./fig_os/fig02.png}}
\caption{Depths of the model vertical $\sigma$-levels along a section of the Irminger Basin for (a) the 6-km simulation (NATL), and (b) the 2-km simulation (POLGYR).}
\label{vertical_level}
\end{figure} 

\begin{figure}[t]
\centerline{\includegraphics[width=18cm]{./fig_os/fig03.png}}
\caption{Mean velocity averaged over 2002-2008 at the surface (left) and 1000-m (right) in NATL (a,b), POLGYR (c,d) and observations, NOAA drifters and ANDRO (e,f).}
\label{vel}
\end{figure}

\begin{figure}[t]
\centerline{\includegraphics[width=15cm]{./fig_os/fig04.png}}
\caption{Time mean (a) barotropic streamfunction and (b) mean barotropic vorticity.}
\label{psil}
\end{figure}

\begin{figure}[t]
\centerline{\includegraphics[width=18cm]{./fig_os/fig05.png}}
\caption{Mean Surface Eddy Kinetic Energy (left) and Mean Eddy Available Potential Energy between 2002 and 2008 (right) in NATL (a,b), POLGYR (c,d). There are compared with result from the NOAA database (e) and the EAPE Atlas from Roullet$\&$al (f)}
\label{energy}
\end{figure}

\begin{figure}[t]
\centerline{\includegraphics[width=18cm]{./fig_os/fig06.png}}
\caption{Time mean (a) bottom pressure torque, (b) non-linear terms, and (c) sum of the two for Eastern Greenland in the 2-km North Atlantic subpolar gyre simulation.}
\label{BPT_NL_unsmoothed}
\end{figure} 

\begin{figure}[t]
\centerline{\includegraphics[width=18cm]{./fig_os/fig07.png}}
\caption{ Time mean of the planetary vorticity (a), bottom pressure torque (b), non linear terms (c), wind stress curl (d), and bottom drag curl (e). As bottom pressure torque and non-linear terms are canceling each other their sum is plotted in (f). The fields have been smoothed using a kernel of 1$^{\circ}$ radius. The blue contour represents the limit of our shelf area and is the -3 Sv barotropic streamline. }
\label{spatial_BV}
\end{figure} 

\begin{figure}[t]
\centerline{\includegraphics[width=20cm]{./fig_os/fig08.png}}
\caption{(a) Bottom Pressure Torque, (b) $-fw_b$, (c)$\frac{f}{h} \overline{u_g} \cdot \nabla H$, and (d) $H \; JEBAR$ for the 2-km North Atlantic subpolar gyre simulation smoothed with a 25 km Gaussian Kernel.}
\label{decomp_bpt}
\end{figure} 

\begin{figure}[t]
\centerline{\includegraphics[width=18cm]{./fig_os/fig09.png}}
\caption{Integration of the barotropic vorticity terms over the SPG including or excluding the shelf area (respectively (a) and (c)). The Subpolar gyre area without the shelf corresponds to the -3 Sv contour. The shelf balance is plotted in (b). }
\label{intgyre_BV}
\end{figure} 

\begin{figure}[t]
\centerline{\includegraphics[width=15cm]{./fig_os/fig10.png}}
\caption{Integration of the barotropic vorticity terms in the slope area (a, defined between the barotropic streamfunction contour -3 Sv and the 3000-m isobath) and interior (b).}
\label{int_bound_inner_BV}
\end{figure} 

\begin{figure}[t]
\centerline{\includegraphics[width=18cm]{./fig_os/fig11.png}}
\caption{Barotropic vorticity balance integrated over different parts of the gyre along the slope.}
\label{bv_zones}
\end{figure}

\begin{figure}[t]
\centerline{\includegraphics[width=12cm]{./fig_os/fig12.png}}
\caption{Vertical section of the mean along-stream current near Iceland shelf (a), Eastern Reykjanes Ridge (b), Denmark Strait (c), Eastern Greenland (d), Northern Labrador Current (e), and Southern Labrador Current (f). Red solid lines and green dashed lines are velocity and isopycnal contours, respectively, while the black dashed line is the limit of integration near the shelf. The black contour on the topography map represents the area on which barotropic vorticity terms are integrated.}
\label{section_vit}
\end{figure} 

\begin{figure}[t]
\centerline{\includegraphics[width=18cm]{./fig_os/fig13.png}}
\caption{Integration of the nonlinear term over the slope (c, in blue) and interior areas (c, in green) for the mean-eddy decomposition (a) and the barotropic-baroclinic decomposition (b). The hatches are highlighting the contribution from the NorthWest Corner.}
\label{int_NL}
\end{figure} 


%\begin{figure}[t]
%\centerline{\includegraphics[width=18cm]{./v_b/test_scheme.png}}
%\caption{Schematic of the barotropic vorticity sources in the SubPolar Gyre}
%\label{scheme}
%\end{figure} 





\end{document} 